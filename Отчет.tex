\documentclass[12pt,a4paper]{scrartcl}
\usepackage[utf8]{inputenc}
\usepackage[english,russian]{babel}
\usepackage{indentfirst}
\usepackage{misccorr}
\usepackage{graphicx}
\usepackage{amsmath}

\begin{document}

\begin{center}
\begin{large}
Проверка выполнимости модели физического маятника
\end{large}
	\bigskip
     
Ахундзянов Амир Андреевич \\
Цветков Петр Алексеевич 
\end{center}

\section{Задачи и цели}
В работе с помощью датчика, фиксирующего угол поворота, и тел с разными моментами инерции и разными моментами сил в диапазоне малых отклонений от устойчивого равновесия мы сравнили экспериментальные значения периода колебания с теоретическими. Также был экспериментально оценён диапазон отклонений в котором модель выполняется.

\section{Теоретическое обоснование}

\section{Методика и оборудование}

\section{Результаты измерений и обработка данных}


\end{document}
